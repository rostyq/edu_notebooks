
% Default to the notebook output style

    


% Inherit from the specified cell style.




    
\documentclass[11pt]{article}

    
    
    \usepackage[T1]{fontenc}
    % Nicer default font (+ math font) than Computer Modern for most use cases
    \usepackage{mathpazo}

    % Basic figure setup, for now with no caption control since it's done
    % automatically by Pandoc (which extracts ![](path) syntax from Markdown).
    \usepackage{graphicx}
    % We will generate all images so they have a width \maxwidth. This means
    % that they will get their normal width if they fit onto the page, but
    % are scaled down if they would overflow the margins.
    \makeatletter
    \def\maxwidth{\ifdim\Gin@nat@width>\linewidth\linewidth
    \else\Gin@nat@width\fi}
    \makeatother
    \let\Oldincludegraphics\includegraphics
    % Set max figure width to be 80% of text width, for now hardcoded.
    \renewcommand{\includegraphics}[1]{\Oldincludegraphics[width=.8\maxwidth]{#1}}
    % Ensure that by default, figures have no caption (until we provide a
    % proper Figure object with a Caption API and a way to capture that
    % in the conversion process - todo).
    \usepackage{caption}
    \DeclareCaptionLabelFormat{nolabel}{}
    \captionsetup{labelformat=nolabel}

    \usepackage{adjustbox} % Used to constrain images to a maximum size 
    \usepackage{xcolor} % Allow colors to be defined
    \usepackage{enumerate} % Needed for markdown enumerations to work
    \usepackage{geometry} % Used to adjust the document margins
    \usepackage{amsmath} % Equations
    \usepackage{amssymb} % Equations
    \usepackage{textcomp} % defines textquotesingle
    % Hack from http://tex.stackexchange.com/a/47451/13684:
    \AtBeginDocument{%
        \def\PYZsq{\textquotesingle}% Upright quotes in Pygmentized code
    }
    \usepackage{upquote} % Upright quotes for verbatim code
    \usepackage{eurosym} % defines \euro
    \usepackage[mathletters]{ucs} % Extended unicode (utf-8) support
    \usepackage[utf8x]{inputenc} % Allow utf-8 characters in the tex document
    \usepackage{fancyvrb} % verbatim replacement that allows latex
    \usepackage{grffile} % extends the file name processing of package graphics 
                         % to support a larger range 
    % The hyperref package gives us a pdf with properly built
    % internal navigation ('pdf bookmarks' for the table of contents,
    % internal cross-reference links, web links for URLs, etc.)
    \usepackage{hyperref}
    \usepackage{longtable} % longtable support required by pandoc >1.10
    \usepackage{booktabs}  % table support for pandoc > 1.12.2
    \usepackage[inline]{enumitem} % IRkernel/repr support (it uses the enumerate* environment)
    \usepackage[normalem]{ulem} % ulem is needed to support strikethroughs (\sout)
                                % normalem makes italics be italics, not underlines
    

    
    
    % Colors for the hyperref package
    \definecolor{urlcolor}{rgb}{0,.145,.698}
    \definecolor{linkcolor}{rgb}{.71,0.21,0.01}
    \definecolor{citecolor}{rgb}{.12,.54,.11}

    % ANSI colors
    \definecolor{ansi-black}{HTML}{3E424D}
    \definecolor{ansi-black-intense}{HTML}{282C36}
    \definecolor{ansi-red}{HTML}{E75C58}
    \definecolor{ansi-red-intense}{HTML}{B22B31}
    \definecolor{ansi-green}{HTML}{00A250}
    \definecolor{ansi-green-intense}{HTML}{007427}
    \definecolor{ansi-yellow}{HTML}{DDB62B}
    \definecolor{ansi-yellow-intense}{HTML}{B27D12}
    \definecolor{ansi-blue}{HTML}{208FFB}
    \definecolor{ansi-blue-intense}{HTML}{0065CA}
    \definecolor{ansi-magenta}{HTML}{D160C4}
    \definecolor{ansi-magenta-intense}{HTML}{A03196}
    \definecolor{ansi-cyan}{HTML}{60C6C8}
    \definecolor{ansi-cyan-intense}{HTML}{258F8F}
    \definecolor{ansi-white}{HTML}{C5C1B4}
    \definecolor{ansi-white-intense}{HTML}{A1A6B2}

    % commands and environments needed by pandoc snippets
    % extracted from the output of `pandoc -s`
    \providecommand{\tightlist}{%
      \setlength{\itemsep}{0pt}\setlength{\parskip}{0pt}}
    \DefineVerbatimEnvironment{Highlighting}{Verbatim}{commandchars=\\\{\}}
    % Add ',fontsize=\small' for more characters per line
    \newenvironment{Shaded}{}{}
    \newcommand{\KeywordTok}[1]{\textcolor[rgb]{0.00,0.44,0.13}{\textbf{{#1}}}}
    \newcommand{\DataTypeTok}[1]{\textcolor[rgb]{0.56,0.13,0.00}{{#1}}}
    \newcommand{\DecValTok}[1]{\textcolor[rgb]{0.25,0.63,0.44}{{#1}}}
    \newcommand{\BaseNTok}[1]{\textcolor[rgb]{0.25,0.63,0.44}{{#1}}}
    \newcommand{\FloatTok}[1]{\textcolor[rgb]{0.25,0.63,0.44}{{#1}}}
    \newcommand{\CharTok}[1]{\textcolor[rgb]{0.25,0.44,0.63}{{#1}}}
    \newcommand{\StringTok}[1]{\textcolor[rgb]{0.25,0.44,0.63}{{#1}}}
    \newcommand{\CommentTok}[1]{\textcolor[rgb]{0.38,0.63,0.69}{\textit{{#1}}}}
    \newcommand{\OtherTok}[1]{\textcolor[rgb]{0.00,0.44,0.13}{{#1}}}
    \newcommand{\AlertTok}[1]{\textcolor[rgb]{1.00,0.00,0.00}{\textbf{{#1}}}}
    \newcommand{\FunctionTok}[1]{\textcolor[rgb]{0.02,0.16,0.49}{{#1}}}
    \newcommand{\RegionMarkerTok}[1]{{#1}}
    \newcommand{\ErrorTok}[1]{\textcolor[rgb]{1.00,0.00,0.00}{\textbf{{#1}}}}
    \newcommand{\NormalTok}[1]{{#1}}
    
    % Additional commands for more recent versions of Pandoc
    \newcommand{\ConstantTok}[1]{\textcolor[rgb]{0.53,0.00,0.00}{{#1}}}
    \newcommand{\SpecialCharTok}[1]{\textcolor[rgb]{0.25,0.44,0.63}{{#1}}}
    \newcommand{\VerbatimStringTok}[1]{\textcolor[rgb]{0.25,0.44,0.63}{{#1}}}
    \newcommand{\SpecialStringTok}[1]{\textcolor[rgb]{0.73,0.40,0.53}{{#1}}}
    \newcommand{\ImportTok}[1]{{#1}}
    \newcommand{\DocumentationTok}[1]{\textcolor[rgb]{0.73,0.13,0.13}{\textit{{#1}}}}
    \newcommand{\AnnotationTok}[1]{\textcolor[rgb]{0.38,0.63,0.69}{\textbf{\textit{{#1}}}}}
    \newcommand{\CommentVarTok}[1]{\textcolor[rgb]{0.38,0.63,0.69}{\textbf{\textit{{#1}}}}}
    \newcommand{\VariableTok}[1]{\textcolor[rgb]{0.10,0.09,0.49}{{#1}}}
    \newcommand{\ControlFlowTok}[1]{\textcolor[rgb]{0.00,0.44,0.13}{\textbf{{#1}}}}
    \newcommand{\OperatorTok}[1]{\textcolor[rgb]{0.40,0.40,0.40}{{#1}}}
    \newcommand{\BuiltInTok}[1]{{#1}}
    \newcommand{\ExtensionTok}[1]{{#1}}
    \newcommand{\PreprocessorTok}[1]{\textcolor[rgb]{0.74,0.48,0.00}{{#1}}}
    \newcommand{\AttributeTok}[1]{\textcolor[rgb]{0.49,0.56,0.16}{{#1}}}
    \newcommand{\InformationTok}[1]{\textcolor[rgb]{0.38,0.63,0.69}{\textbf{\textit{{#1}}}}}
    \newcommand{\WarningTok}[1]{\textcolor[rgb]{0.38,0.63,0.69}{\textbf{\textit{{#1}}}}}
    
    
    % Define a nice break command that doesn't care if a line doesn't already
    % exist.
    \def\br{\hspace*{\fill} \\* }
    % Math Jax compatability definitions
    \def\gt{>}
    \def\lt{<}
    % Document parameters
    \title{calc\_work}
    
    
    

    % Pygments definitions
    
\makeatletter
\def\PY@reset{\let\PY@it=\relax \let\PY@bf=\relax%
    \let\PY@ul=\relax \let\PY@tc=\relax%
    \let\PY@bc=\relax \let\PY@ff=\relax}
\def\PY@tok#1{\csname PY@tok@#1\endcsname}
\def\PY@toks#1+{\ifx\relax#1\empty\else%
    \PY@tok{#1}\expandafter\PY@toks\fi}
\def\PY@do#1{\PY@bc{\PY@tc{\PY@ul{%
    \PY@it{\PY@bf{\PY@ff{#1}}}}}}}
\def\PY#1#2{\PY@reset\PY@toks#1+\relax+\PY@do{#2}}

\expandafter\def\csname PY@tok@w\endcsname{\def\PY@tc##1{\textcolor[rgb]{0.73,0.73,0.73}{##1}}}
\expandafter\def\csname PY@tok@c\endcsname{\let\PY@it=\textit\def\PY@tc##1{\textcolor[rgb]{0.25,0.50,0.50}{##1}}}
\expandafter\def\csname PY@tok@cp\endcsname{\def\PY@tc##1{\textcolor[rgb]{0.74,0.48,0.00}{##1}}}
\expandafter\def\csname PY@tok@k\endcsname{\let\PY@bf=\textbf\def\PY@tc##1{\textcolor[rgb]{0.00,0.50,0.00}{##1}}}
\expandafter\def\csname PY@tok@kp\endcsname{\def\PY@tc##1{\textcolor[rgb]{0.00,0.50,0.00}{##1}}}
\expandafter\def\csname PY@tok@kt\endcsname{\def\PY@tc##1{\textcolor[rgb]{0.69,0.00,0.25}{##1}}}
\expandafter\def\csname PY@tok@o\endcsname{\def\PY@tc##1{\textcolor[rgb]{0.40,0.40,0.40}{##1}}}
\expandafter\def\csname PY@tok@ow\endcsname{\let\PY@bf=\textbf\def\PY@tc##1{\textcolor[rgb]{0.67,0.13,1.00}{##1}}}
\expandafter\def\csname PY@tok@nb\endcsname{\def\PY@tc##1{\textcolor[rgb]{0.00,0.50,0.00}{##1}}}
\expandafter\def\csname PY@tok@nf\endcsname{\def\PY@tc##1{\textcolor[rgb]{0.00,0.00,1.00}{##1}}}
\expandafter\def\csname PY@tok@nc\endcsname{\let\PY@bf=\textbf\def\PY@tc##1{\textcolor[rgb]{0.00,0.00,1.00}{##1}}}
\expandafter\def\csname PY@tok@nn\endcsname{\let\PY@bf=\textbf\def\PY@tc##1{\textcolor[rgb]{0.00,0.00,1.00}{##1}}}
\expandafter\def\csname PY@tok@ne\endcsname{\let\PY@bf=\textbf\def\PY@tc##1{\textcolor[rgb]{0.82,0.25,0.23}{##1}}}
\expandafter\def\csname PY@tok@nv\endcsname{\def\PY@tc##1{\textcolor[rgb]{0.10,0.09,0.49}{##1}}}
\expandafter\def\csname PY@tok@no\endcsname{\def\PY@tc##1{\textcolor[rgb]{0.53,0.00,0.00}{##1}}}
\expandafter\def\csname PY@tok@nl\endcsname{\def\PY@tc##1{\textcolor[rgb]{0.63,0.63,0.00}{##1}}}
\expandafter\def\csname PY@tok@ni\endcsname{\let\PY@bf=\textbf\def\PY@tc##1{\textcolor[rgb]{0.60,0.60,0.60}{##1}}}
\expandafter\def\csname PY@tok@na\endcsname{\def\PY@tc##1{\textcolor[rgb]{0.49,0.56,0.16}{##1}}}
\expandafter\def\csname PY@tok@nt\endcsname{\let\PY@bf=\textbf\def\PY@tc##1{\textcolor[rgb]{0.00,0.50,0.00}{##1}}}
\expandafter\def\csname PY@tok@nd\endcsname{\def\PY@tc##1{\textcolor[rgb]{0.67,0.13,1.00}{##1}}}
\expandafter\def\csname PY@tok@s\endcsname{\def\PY@tc##1{\textcolor[rgb]{0.73,0.13,0.13}{##1}}}
\expandafter\def\csname PY@tok@sd\endcsname{\let\PY@it=\textit\def\PY@tc##1{\textcolor[rgb]{0.73,0.13,0.13}{##1}}}
\expandafter\def\csname PY@tok@si\endcsname{\let\PY@bf=\textbf\def\PY@tc##1{\textcolor[rgb]{0.73,0.40,0.53}{##1}}}
\expandafter\def\csname PY@tok@se\endcsname{\let\PY@bf=\textbf\def\PY@tc##1{\textcolor[rgb]{0.73,0.40,0.13}{##1}}}
\expandafter\def\csname PY@tok@sr\endcsname{\def\PY@tc##1{\textcolor[rgb]{0.73,0.40,0.53}{##1}}}
\expandafter\def\csname PY@tok@ss\endcsname{\def\PY@tc##1{\textcolor[rgb]{0.10,0.09,0.49}{##1}}}
\expandafter\def\csname PY@tok@sx\endcsname{\def\PY@tc##1{\textcolor[rgb]{0.00,0.50,0.00}{##1}}}
\expandafter\def\csname PY@tok@m\endcsname{\def\PY@tc##1{\textcolor[rgb]{0.40,0.40,0.40}{##1}}}
\expandafter\def\csname PY@tok@gh\endcsname{\let\PY@bf=\textbf\def\PY@tc##1{\textcolor[rgb]{0.00,0.00,0.50}{##1}}}
\expandafter\def\csname PY@tok@gu\endcsname{\let\PY@bf=\textbf\def\PY@tc##1{\textcolor[rgb]{0.50,0.00,0.50}{##1}}}
\expandafter\def\csname PY@tok@gd\endcsname{\def\PY@tc##1{\textcolor[rgb]{0.63,0.00,0.00}{##1}}}
\expandafter\def\csname PY@tok@gi\endcsname{\def\PY@tc##1{\textcolor[rgb]{0.00,0.63,0.00}{##1}}}
\expandafter\def\csname PY@tok@gr\endcsname{\def\PY@tc##1{\textcolor[rgb]{1.00,0.00,0.00}{##1}}}
\expandafter\def\csname PY@tok@ge\endcsname{\let\PY@it=\textit}
\expandafter\def\csname PY@tok@gs\endcsname{\let\PY@bf=\textbf}
\expandafter\def\csname PY@tok@gp\endcsname{\let\PY@bf=\textbf\def\PY@tc##1{\textcolor[rgb]{0.00,0.00,0.50}{##1}}}
\expandafter\def\csname PY@tok@go\endcsname{\def\PY@tc##1{\textcolor[rgb]{0.53,0.53,0.53}{##1}}}
\expandafter\def\csname PY@tok@gt\endcsname{\def\PY@tc##1{\textcolor[rgb]{0.00,0.27,0.87}{##1}}}
\expandafter\def\csname PY@tok@err\endcsname{\def\PY@bc##1{\setlength{\fboxsep}{0pt}\fcolorbox[rgb]{1.00,0.00,0.00}{1,1,1}{\strut ##1}}}
\expandafter\def\csname PY@tok@kc\endcsname{\let\PY@bf=\textbf\def\PY@tc##1{\textcolor[rgb]{0.00,0.50,0.00}{##1}}}
\expandafter\def\csname PY@tok@kd\endcsname{\let\PY@bf=\textbf\def\PY@tc##1{\textcolor[rgb]{0.00,0.50,0.00}{##1}}}
\expandafter\def\csname PY@tok@kn\endcsname{\let\PY@bf=\textbf\def\PY@tc##1{\textcolor[rgb]{0.00,0.50,0.00}{##1}}}
\expandafter\def\csname PY@tok@kr\endcsname{\let\PY@bf=\textbf\def\PY@tc##1{\textcolor[rgb]{0.00,0.50,0.00}{##1}}}
\expandafter\def\csname PY@tok@bp\endcsname{\def\PY@tc##1{\textcolor[rgb]{0.00,0.50,0.00}{##1}}}
\expandafter\def\csname PY@tok@fm\endcsname{\def\PY@tc##1{\textcolor[rgb]{0.00,0.00,1.00}{##1}}}
\expandafter\def\csname PY@tok@vc\endcsname{\def\PY@tc##1{\textcolor[rgb]{0.10,0.09,0.49}{##1}}}
\expandafter\def\csname PY@tok@vg\endcsname{\def\PY@tc##1{\textcolor[rgb]{0.10,0.09,0.49}{##1}}}
\expandafter\def\csname PY@tok@vi\endcsname{\def\PY@tc##1{\textcolor[rgb]{0.10,0.09,0.49}{##1}}}
\expandafter\def\csname PY@tok@vm\endcsname{\def\PY@tc##1{\textcolor[rgb]{0.10,0.09,0.49}{##1}}}
\expandafter\def\csname PY@tok@sa\endcsname{\def\PY@tc##1{\textcolor[rgb]{0.73,0.13,0.13}{##1}}}
\expandafter\def\csname PY@tok@sb\endcsname{\def\PY@tc##1{\textcolor[rgb]{0.73,0.13,0.13}{##1}}}
\expandafter\def\csname PY@tok@sc\endcsname{\def\PY@tc##1{\textcolor[rgb]{0.73,0.13,0.13}{##1}}}
\expandafter\def\csname PY@tok@dl\endcsname{\def\PY@tc##1{\textcolor[rgb]{0.73,0.13,0.13}{##1}}}
\expandafter\def\csname PY@tok@s2\endcsname{\def\PY@tc##1{\textcolor[rgb]{0.73,0.13,0.13}{##1}}}
\expandafter\def\csname PY@tok@sh\endcsname{\def\PY@tc##1{\textcolor[rgb]{0.73,0.13,0.13}{##1}}}
\expandafter\def\csname PY@tok@s1\endcsname{\def\PY@tc##1{\textcolor[rgb]{0.73,0.13,0.13}{##1}}}
\expandafter\def\csname PY@tok@mb\endcsname{\def\PY@tc##1{\textcolor[rgb]{0.40,0.40,0.40}{##1}}}
\expandafter\def\csname PY@tok@mf\endcsname{\def\PY@tc##1{\textcolor[rgb]{0.40,0.40,0.40}{##1}}}
\expandafter\def\csname PY@tok@mh\endcsname{\def\PY@tc##1{\textcolor[rgb]{0.40,0.40,0.40}{##1}}}
\expandafter\def\csname PY@tok@mi\endcsname{\def\PY@tc##1{\textcolor[rgb]{0.40,0.40,0.40}{##1}}}
\expandafter\def\csname PY@tok@il\endcsname{\def\PY@tc##1{\textcolor[rgb]{0.40,0.40,0.40}{##1}}}
\expandafter\def\csname PY@tok@mo\endcsname{\def\PY@tc##1{\textcolor[rgb]{0.40,0.40,0.40}{##1}}}
\expandafter\def\csname PY@tok@ch\endcsname{\let\PY@it=\textit\def\PY@tc##1{\textcolor[rgb]{0.25,0.50,0.50}{##1}}}
\expandafter\def\csname PY@tok@cm\endcsname{\let\PY@it=\textit\def\PY@tc##1{\textcolor[rgb]{0.25,0.50,0.50}{##1}}}
\expandafter\def\csname PY@tok@cpf\endcsname{\let\PY@it=\textit\def\PY@tc##1{\textcolor[rgb]{0.25,0.50,0.50}{##1}}}
\expandafter\def\csname PY@tok@c1\endcsname{\let\PY@it=\textit\def\PY@tc##1{\textcolor[rgb]{0.25,0.50,0.50}{##1}}}
\expandafter\def\csname PY@tok@cs\endcsname{\let\PY@it=\textit\def\PY@tc##1{\textcolor[rgb]{0.25,0.50,0.50}{##1}}}

\def\PYZbs{\char`\\}
\def\PYZus{\char`\_}
\def\PYZob{\char`\{}
\def\PYZcb{\char`\}}
\def\PYZca{\char`\^}
\def\PYZam{\char`\&}
\def\PYZlt{\char`\<}
\def\PYZgt{\char`\>}
\def\PYZsh{\char`\#}
\def\PYZpc{\char`\%}
\def\PYZdl{\char`\$}
\def\PYZhy{\char`\-}
\def\PYZsq{\char`\'}
\def\PYZdq{\char`\"}
\def\PYZti{\char`\~}
% for compatibility with earlier versions
\def\PYZat{@}
\def\PYZlb{[}
\def\PYZrb{]}
\makeatother


    % Exact colors from NB
    \definecolor{incolor}{rgb}{0.0, 0.0, 0.5}
    \definecolor{outcolor}{rgb}{0.545, 0.0, 0.0}



    
    % Prevent overflowing lines due to hard-to-break entities
    \sloppy 
    % Setup hyperref package
    \hypersetup{
      breaklinks=true,  % so long urls are correctly broken across lines
      colorlinks=true,
      urlcolor=urlcolor,
      linkcolor=linkcolor,
      citecolor=citecolor,
      }
    % Slightly bigger margins than the latex defaults
    
    \geometry{verbose,tmargin=1in,bmargin=1in,lmargin=1in,rmargin=1in}
    
    

    \begin{document}
    
    
    \maketitle
    
    

    
    \section{РГР: Визначення температурних полів та параметрів масопереносу
при імпульсному
впливі}\label{ux440ux433ux440-ux432ux438ux437ux43dux430ux447ux435ux43dux43dux44f-ux442ux435ux43cux43fux435ux440ux430ux442ux443ux440ux43dux438ux445-ux43fux43eux43bux456ux432-ux442ux430-ux43fux430ux440ux430ux43cux435ux442ux440ux456ux432-ux43cux430ux441ux43eux43fux435ux440ux435ux43dux43eux441ux443-ux43fux440ux438-ux456ux43cux43fux443ux43bux44cux441ux43dux43eux43cux443-ux432ux43fux43bux438ux432ux456}

    \begin{Verbatim}[commandchars=\\\{\}]
{\color{incolor}In [{\color{incolor}1}]:} \PY{k+kn}{import} \PY{n+nn}{pandas} \PY{k}{as} \PY{n+nn}{pd}
        \PY{k+kn}{import} \PY{n+nn}{numpy} \PY{k}{as} \PY{n+nn}{np}
        \PY{k+kn}{from} \PY{n+nn}{IPython}\PY{n+nn}{.}\PY{n+nn}{display} \PY{k}{import} \PY{n}{Latex}
\end{Verbatim}


    \subsection{Розрахунок температурних
полів}\label{ux440ux43eux437ux440ux430ux445ux443ux43dux43eux43a-ux442ux435ux43cux43fux435ux440ux430ux442ux443ux440ux43dux438ux445-ux43fux43eux43bux456ux432}

    **Визначити за даними таблиці значення \(\lambda\) {[}Дж/см*Kc{]} для
елементів Cr, Cu, Ni.**

\[\lambda_i = a_i \gamma_i c_i,\] де:

\begin{itemize}
\tightlist
\item
  \(\lambda_i\) - thermal conductivity
\item
  \(c_i\) - thermal capacity
\item
  \(a_i\) - thermal diffusivity
\item
  \(\gamma_i\) - density
\end{itemize}

    \begin{Verbatim}[commandchars=\\\{\}]
{\color{incolor}In [{\color{incolor}2}]:} \PY{n}{df\PYZus{}in} \PY{o}{=} \PY{n}{pd}\PY{o}{.}\PY{n}{DataFrame}\PY{p}{(}
            \PY{p}{\PYZob{}}
                \PY{l+s+s1}{\PYZsq{}}\PY{l+s+s1}{thermal\PYZus{}capacity}\PY{l+s+s1}{\PYZsq{}}\PY{p}{:}    \PY{p}{[}\PY{l+m+mf}{0.56}\PY{p}{,} \PY{l+m+mf}{0.47}\PY{p}{,} \PY{l+m+mf}{0.58}\PY{p}{]}\PY{p}{,}
                \PY{l+s+s1}{\PYZsq{}}\PY{l+s+s1}{thermal\PYZus{}diffusivity}\PY{l+s+s1}{\PYZsq{}}\PY{p}{:} \PY{p}{[}\PY{l+m+mf}{0.14}\PY{p}{,} \PY{l+m+mf}{0.85}\PY{p}{,} \PY{l+m+mf}{0.16}\PY{p}{]}\PY{p}{,}
                \PY{l+s+s1}{\PYZsq{}}\PY{l+s+s1}{density}\PY{l+s+s1}{\PYZsq{}}\PY{p}{:}             \PY{p}{[}\PY{l+m+mf}{8.9}\PY{p}{,}  \PY{l+m+mf}{8.96}\PY{p}{,} \PY{l+m+mf}{7.1}\PY{p}{]}\PY{p}{,}
                \PY{l+s+s1}{\PYZsq{}}\PY{l+s+s1}{thickness}\PY{l+s+s1}{\PYZsq{}}\PY{p}{:}           \PY{p}{[}\PY{l+m+mi}{100}\PY{p}{]}\PY{o}{*}\PY{l+m+mi}{3}
            \PY{p}{\PYZcb{}}\PY{p}{,}
            \PY{n}{index}\PY{o}{=}\PY{p}{[}\PY{l+s+s1}{\PYZsq{}}\PY{l+s+s1}{Ni}\PY{l+s+s1}{\PYZsq{}}\PY{p}{,} \PY{l+s+s1}{\PYZsq{}}\PY{l+s+s1}{Cu}\PY{l+s+s1}{\PYZsq{}}\PY{p}{,} \PY{l+s+s1}{\PYZsq{}}\PY{l+s+s1}{Cr}\PY{l+s+s1}{\PYZsq{}}\PY{p}{]}
        \PY{p}{)}
        \PY{n}{df\PYZus{}in}\PY{p}{[}\PY{l+s+s1}{\PYZsq{}}\PY{l+s+s1}{thermal\PYZus{}conductivity}\PY{l+s+s1}{\PYZsq{}}\PY{p}{]} \PY{o}{=} \PY{n}{df\PYZus{}in}\PY{o}{.}\PY{n}{thermal\PYZus{}diffusivity} \PY{o}{*} \PY{n}{df\PYZus{}in}\PY{o}{.}\PY{n}{density} \PY{o}{*} \PY{n}{df\PYZus{}in}\PY{o}{.}\PY{n}{thermal\PYZus{}capacity}
        \PY{n}{df\PYZus{}in}
\end{Verbatim}


\begin{Verbatim}[commandchars=\\\{\}]
{\color{outcolor}Out[{\color{outcolor}2}]:}     density  thermal\_capacity  thermal\_diffusivity  thickness  \textbackslash{}
        Ni     8.90              0.56                 0.14        100   
        Cu     8.96              0.47                 0.85        100   
        Cr     7.10              0.58                 0.16        100   
        
            thermal\_conductivity  
        Ni               0.69776  
        Cu               3.57952  
        Cr               0.65888  
\end{Verbatim}
            
    \textbf{Розрахувати загальну теплопровідність \(\bar{\lambda}\) системи
Cr-Cu-Ni:}

\[\bar{\lambda} = \frac{d_1\lambda_1 + d_2\lambda_2 + d_3\lambda_3}{d}\]

    \begin{Verbatim}[commandchars=\\\{\}]
{\color{incolor}In [{\color{incolor}3}]:} \PY{n}{sys\PYZus{}thermal\PYZus{}conductivity} \PY{o}{=} \PY{p}{(}\PY{n}{df\PYZus{}in}\PY{o}{.}\PY{n}{thermal\PYZus{}conductivity} \PY{o}{*} \PY{n}{df\PYZus{}in}\PY{o}{.}\PY{n}{thickness}\PY{p}{)}\PY{o}{.}\PY{n}{sum}\PY{p}{(}\PY{p}{)} \PY{o}{/} \PY{n}{df\PYZus{}in}\PY{o}{.}\PY{n}{thickness}\PY{o}{.}\PY{n}{sum}\PY{p}{(}\PY{p}{)}
        \PY{n}{Latex}\PY{p}{(}\PY{l+s+sa}{r}\PY{l+s+s1}{\PYZsq{}}\PY{l+s+s1}{\PYZdl{}}\PY{l+s+s1}{\PYZbs{}}\PY{l+s+s1}{bar}\PY{l+s+s1}{\PYZob{}}\PY{l+s+s1}{\PYZbs{}}\PY{l+s+s1}{lambda\PYZcb{} = }\PY{l+s+si}{\PYZpc{}.2f}\PY{l+s+s1}{\PYZdl{}}\PY{l+s+s1}{\PYZsq{}} \PY{o}{\PYZpc{}} \PY{n}{sys\PYZus{}thermal\PYZus{}conductivity}\PY{p}{)}
\end{Verbatim}

\texttt{\color{outcolor}Out[{\color{outcolor}3}]:}
    
    $\bar{\lambda} = 1.65$

    

    \textbf{Розрахувати ефективне значення температуропровідності
\(a_{\textit{пл}}\) плівкової структури уздовж поверхні за формулою:}

\[a_{\textit{пл}} = \frac{\bar{\lambda}}{\bar{c}\bar{\gamma}}\]

    \begin{Verbatim}[commandchars=\\\{\}]
{\color{incolor}In [{\color{incolor}4}]:} \PY{n}{sys\PYZus{}thermal\PYZus{}capacity} \PY{o}{=} \PY{p}{(}\PY{n}{df\PYZus{}in}\PY{o}{.}\PY{n}{thermal\PYZus{}capacity} \PY{o}{*} \PY{n}{df\PYZus{}in}\PY{o}{.}\PY{n}{thickness}\PY{p}{)}\PY{o}{.}\PY{n}{sum}\PY{p}{(}\PY{p}{)} \PY{o}{/} \PY{n}{df\PYZus{}in}\PY{o}{.}\PY{n}{thickness}\PY{o}{.}\PY{n}{sum}\PY{p}{(}\PY{p}{)}
        \PY{n}{sys\PYZus{}density} \PY{o}{=} \PY{p}{(}\PY{n}{df\PYZus{}in}\PY{o}{.}\PY{n}{density} \PY{o}{*} \PY{n}{df\PYZus{}in}\PY{o}{.}\PY{n}{thickness}\PY{p}{)}\PY{o}{.}\PY{n}{sum}\PY{p}{(}\PY{p}{)} \PY{o}{/} \PY{n}{df\PYZus{}in}\PY{o}{.}\PY{n}{thickness}\PY{o}{.}\PY{n}{sum}\PY{p}{(}\PY{p}{)}
        \PY{n}{efficient\PYZus{}thermal\PYZus{}diffusivity} \PY{o}{=} \PY{n}{sys\PYZus{}thermal\PYZus{}conductivity} \PY{o}{/} \PY{p}{(}\PY{n}{sys\PYZus{}thermal\PYZus{}capacity} \PY{o}{*} \PY{n}{sys\PYZus{}density}\PY{p}{)}
        \PY{n}{Latex}\PY{p}{(}\PY{l+s+sa}{r}\PY{l+s+s1}{\PYZsq{}}\PY{l+s+s1}{\PYZdl{}a\PYZus{}}\PY{l+s+s1}{\PYZob{}}\PY{l+s+s1}{\PYZbs{}}\PY{l+s+s1}{textit}\PY{l+s+si}{\PYZob{}пл\PYZcb{}}\PY{l+s+s1}{\PYZcb{} = }\PY{l+s+si}{\PYZpc{}.2f}\PY{l+s+s1}{\PYZdl{}}\PY{l+s+s1}{\PYZsq{}} \PY{o}{\PYZpc{}} \PY{n}{efficient\PYZus{}thermal\PYZus{}diffusivity}\PY{p}{)}
\end{Verbatim}

\texttt{\color{outcolor}Out[{\color{outcolor}4}]:}
    
    $a_{\textit{пл}} = 0.37$

    

    \textbf{Визначити температуру плівкової структури Cr-Cu-Ni в епіцентрі
під дією лазерного імпульсу для 10 часових інтервалів, на які умовно
розбивається тривалість імпульсу, за формулою:}

\[
T = \frac{2E}{\pi^{3/2}\tau}
\frac{1-R}{c\gamma\sqrt{a_c}}
\frac{\sqrt{\Delta t \dot{} n}}{r^2_{\it\Gamma}+ 4a_{\textit{пл}}\Delta t \dot{} n},
\] де:

\begin{itemize}
\tightlist
\item
  \(\tau\) - тривалість імпульсу
\item
  \(\Delta t\) величина, яка дорівнює
  \[\Delta t = \frac{\tau}{n'_{\max}}, \; n'_{\max} = 10\]
\item
  \(t\) - час, протягом якого діє джерело тепла:
  \[t = \Delta t \dot{} n' = \frac{\tau\dot{}n'}{10},\] \(n\) - часовий
  інтервал (\(n' = 1, 2, 3, \dots\))
\end{itemize}

    \begin{itemize}
\tightlist
\item
  Енергія лазерного імпульсу: \(E = 99\,\text{мДж}\)
\item
  \(\frac{1-R}{c\gamma\sqrt{a_c}} = 9.0\,\frac{K\dot{}\text{см}^2\dot{}\text{с}^{1/2}}{\text{Дж}}\)
\item
  Тривалість лазерного імпульсу \(\tau = 2.56\times10^{-3}\,\text{с}\)
\item
  \(r_{\it\Gamma} = 0.05\,\text{см}\)
\end{itemize}

    \begin{Verbatim}[commandchars=\\\{\}]
{\color{incolor}In [{\color{incolor}5}]:} \PY{n}{tau} \PY{o}{=} \PY{l+m+mf}{2.56e\PYZhy{}3}
        \PY{n}{coeff} \PY{o}{=} \PY{l+m+mf}{9.0}
        \PY{n}{E} \PY{o}{=} \PY{l+m+mf}{99e\PYZhy{}3}
        \PY{n}{r\PYZus{}batch} \PY{o}{=} \PY{l+m+mf}{0.05}
        
        \PY{k}{def} \PY{n+nf}{calc\PYZus{}temperature}\PY{p}{(}\PY{n}{tau}\PY{p}{,} \PY{n}{n}\PY{p}{,} \PY{n}{E}\PY{p}{,} \PY{n}{coeff}\PY{p}{,} \PY{n}{r\PYZus{}batch}\PY{p}{,} \PY{n}{efficient\PYZus{}thermal\PYZus{}diffusivity}\PY{p}{)}\PY{p}{:}
            \PY{n}{Delta\PYZus{}t} \PY{o}{=} \PY{n}{tau} \PY{o}{/} \PY{l+m+mi}{10}
            \PY{k}{return} \PY{l+m+mi}{2} \PY{o}{*} \PY{n}{E} \PY{o}{/} \PY{p}{(}\PY{n}{np}\PY{o}{.}\PY{n}{pi}\PY{o}{*}\PY{o}{*}\PY{p}{(}\PY{l+m+mf}{1.5}\PY{p}{)} \PY{o}{*} \PY{n}{tau}\PY{p}{)} \PY{o}{*} \PYZbs{}
                \PY{n}{coeff} \PY{o}{*} \PYZbs{}
                \PY{p}{(}\PY{n}{np}\PY{o}{.}\PY{n}{sqrt}\PY{p}{(}\PY{n}{Delta\PYZus{}t} \PY{o}{*} \PY{n}{n}\PY{p}{)}\PY{p}{)} \PY{o}{/} \PYZbs{}
                \PY{p}{(}\PY{n}{r\PYZus{}batch}\PY{o}{*}\PY{o}{*}\PY{l+m+mi}{2} \PY{o}{+} \PY{l+m+mi}{4}\PY{o}{*}\PY{n}{efficient\PYZus{}thermal\PYZus{}diffusivity} \PY{o}{*} \PY{n}{Delta\PYZus{}t} \PY{o}{*} \PY{n}{n}\PY{p}{)}
        
        \PY{n}{N} \PY{o}{=} \PY{n+nb}{range}\PY{p}{(}\PY{l+m+mi}{1}\PY{p}{,} \PY{l+m+mi}{11}\PY{p}{)}
        \PY{n}{times} \PY{o}{=} \PY{n}{np}\PY{o}{.}\PY{n}{array}\PY{p}{(}\PY{p}{[}\PY{n}{tau} \PY{o}{/} \PY{l+m+mi}{10} \PY{o}{*} \PY{n}{n} \PY{k}{for} \PY{n}{n} \PY{o+ow}{in} \PY{n}{N}\PY{p}{]}\PY{p}{)}
        \PY{n}{temperatures} \PY{o}{=} \PY{n}{np}\PY{o}{.}\PY{n}{array}\PY{p}{(}\PY{p}{[}\PY{n}{calc\PYZus{}temperature}\PY{p}{(}\PY{n}{tau}\PY{p}{,} \PY{n}{n}\PY{p}{,} \PY{n}{E}\PY{p}{,} \PY{n}{coeff}\PY{p}{,} \PY{n}{r\PYZus{}batch}\PY{p}{,} \PY{n}{efficient\PYZus{}thermal\PYZus{}diffusivity}\PY{p}{)} \PY{k}{for} \PY{n}{n} \PY{o+ow}{in} \PY{n}{N}\PY{p}{]}\PY{p}{)}
        
        \PY{n}{df\PYZus{}out} \PY{o}{=} \PY{n}{pd}\PY{o}{.}\PY{n}{DataFrame}\PY{p}{(}\PY{p}{\PYZob{}}\PY{l+s+s1}{\PYZsq{}}\PY{l+s+s1}{time, x10\PYZca{}(\PYZhy{}3) s}\PY{l+s+s1}{\PYZsq{}}\PY{p}{:} \PY{n}{times} \PY{o}{*} \PY{p}{(}\PY{l+m+mi}{10}\PY{o}{*}\PY{o}{*}\PY{l+m+mi}{3}\PY{p}{)}\PY{p}{,} \PY{l+s+s1}{\PYZsq{}}\PY{l+s+s1}{T, K}\PY{l+s+s1}{\PYZsq{}}\PY{p}{:} \PY{n}{temperatures}\PY{p}{\PYZcb{}}\PY{p}{,} \PY{n}{index}\PY{o}{=}\PY{n}{N}\PY{p}{)}
        \PY{n}{df\PYZus{}out}\PY{o}{.}\PY{n}{round}\PY{p}{(}\PY{l+m+mi}{3}\PY{p}{)}
\end{Verbatim}


\begin{Verbatim}[commandchars=\\\{\}]
{\color{outcolor}Out[{\color{outcolor}5}]:}         T, K  time, x10\^{}(-3) s
        1    695.138             0.256
        2    869.096             0.512
        3    953.835             0.768
        4    997.735             1.024
        5   1019.546             1.280
        6   1028.394             1.536
        7   1029.268             1.792
        8   1025.099             2.048
        9   1017.696             2.304
        10  1008.221             2.560
\end{Verbatim}
            
    \subsection{Розрахунок параметрів
масопереносу}\label{ux440ux43eux437ux440ux430ux445ux443ux43dux43eux43a-ux43fux430ux440ux430ux43cux435ux442ux440ux456ux432-ux43cux430ux441ux43eux43fux435ux440ux435ux43dux43eux441ux443}

    \textbf{Визначити середні коефіцієнти дифузії \(D_{\textit{еф}}\) для
міді:} \[D_{\textit{еф}} = \frac{\sum_{j=1}^n D_j t_j}{\tau}\]

Коефіцієнт дифузії розраховується за формулою:

\[D = D_0 \exp{\frac{-E_a}{RT}}\]

    \begin{Verbatim}[commandchars=\\\{\}]
{\color{incolor}In [{\color{incolor}6}]:} \PY{n}{D\PYZus{}0} \PY{o}{=} \PY{l+m+mf}{1e\PYZhy{}3}
        \PY{n}{E} \PY{o}{=} \PY{l+m+mi}{149}
        
        \PY{k}{def} \PY{n+nf}{calc\PYZus{}mass\PYZus{}diffusivity}\PY{p}{(}\PY{n}{T}\PY{p}{,} \PY{n}{D\PYZus{}0}\PY{p}{,} \PY{n}{E}\PY{p}{)}\PY{p}{:}
            \PY{k}{return} \PY{n}{D\PYZus{}0} \PY{o}{*} \PY{n}{np}\PY{o}{.}\PY{n}{exp}\PY{p}{(}\PY{o}{\PYZhy{}}\PY{n}{E}\PY{o}{/}\PY{p}{(}\PY{l+m+mf}{8.31}\PY{o}{*}\PY{n}{T}\PY{p}{)}\PY{p}{)}
        
        \PY{n}{mass\PYZus{}diffusivity} \PY{o}{=} \PY{n}{np}\PY{o}{.}\PY{n}{apply\PYZus{}along\PYZus{}axis}\PY{p}{(}\PY{n}{calc\PYZus{}mass\PYZus{}diffusivity}\PY{p}{,} \PY{n}{axis}\PY{o}{=}\PY{l+m+mi}{0}\PY{p}{,} \PY{n}{arr}\PY{o}{=}\PY{n}{temperatures}\PY{p}{,} \PY{n}{D\PYZus{}0}\PY{o}{=}\PY{n}{D\PYZus{}0}\PY{p}{,} \PY{n}{E}\PY{o}{=}\PY{n}{E}\PY{p}{)}
        \PY{n}{df\PYZus{}out}\PY{p}{[}\PY{l+s+s1}{\PYZsq{}}\PY{l+s+s1}{D}\PY{l+s+s1}{\PYZsq{}}\PY{p}{]} \PY{o}{=} \PY{n}{mass\PYZus{}diffusivity}
        \PY{n}{df\PYZus{}out}
\end{Verbatim}


\begin{Verbatim}[commandchars=\\\{\}]
{\color{outcolor}Out[{\color{outcolor}6}]:}            T, K  time, x10\^{}(-3) s         D
        1    695.137614             0.256  0.000975
        2    869.096409             0.512  0.000980
        3    953.834616             0.768  0.000981
        4    997.734899             1.024  0.000982
        5   1019.546189             1.280  0.000983
        6   1028.394492             1.536  0.000983
        7   1029.268205             1.792  0.000983
        8   1025.098592             2.048  0.000983
        9   1017.696210             2.304  0.000983
        10  1008.220759             2.560  0.000982
\end{Verbatim}
            
    \begin{Verbatim}[commandchars=\\\{\}]
{\color{incolor}In [{\color{incolor}7}]:} \PY{n}{efficient\PYZus{}mass\PYZus{}diffusivity} \PY{o}{=} \PY{n}{np}\PY{o}{.}\PY{n}{sum}\PY{p}{(}\PY{n}{mass\PYZus{}diffusivity} \PY{o}{*} \PY{n}{times}\PY{p}{)} \PY{o}{/} \PY{n}{tau}
        \PY{n}{Latex}\PY{p}{(}\PY{l+s+sa}{r}\PY{l+s+s1}{\PYZsq{}}\PY{l+s+s1}{\PYZdl{}D\PYZus{}}\PY{l+s+s1}{\PYZob{}}\PY{l+s+s1}{\PYZbs{}}\PY{l+s+s1}{textit}\PY{l+s+si}{\PYZob{}еф\PYZcb{}}\PY{l+s+s1}{\PYZcb{} = }\PY{l+s+si}{\PYZpc{}.2f}\PY{l+s+s1}{ }\PY{l+s+s1}{\PYZbs{}}\PY{l+s+s1}{times 10\PYZca{}}\PY{l+s+s1}{\PYZob{}}\PY{l+s+s1}{\PYZhy{}3\PYZcb{}\PYZdl{}}\PY{l+s+s1}{\PYZsq{}} \PY{o}{\PYZpc{}} \PY{p}{(}\PY{n}{efficient\PYZus{}mass\PYZus{}diffusivity} \PY{o}{*} \PY{p}{(}\PY{l+m+mi}{10}\PY{o}{*}\PY{o}{*}\PY{l+m+mi}{3}\PY{p}{)}\PY{p}{)}\PY{p}{)}
\end{Verbatim}

\texttt{\color{outcolor}Out[{\color{outcolor}7}]:}
    
    $D_{\textit{еф}} = 5.40 \times 10^{-3}$

    

    \textbf{Розрахувати середні значення "довжини дифузійного шляху"
\(x = (Dt)^{1/2}\) за значеннями ефективних коефіцієнтів
\(D_{\textit{еф}}\)}

    \begin{Verbatim}[commandchars=\\\{\}]
{\color{incolor}In [{\color{incolor}8}]:} \PY{n}{distance} \PY{o}{=} \PY{p}{(}\PY{n}{efficient\PYZus{}mass\PYZus{}diffusivity} \PY{o}{*} \PY{n}{times}\PY{p}{[}\PY{o}{\PYZhy{}}\PY{l+m+mi}{1}\PY{p}{]}\PY{p}{)} \PY{o}{*}\PY{o}{*} \PY{p}{(}\PY{l+m+mf}{0.5}\PY{p}{)}
        \PY{n}{Latex}\PY{p}{(}\PY{l+s+sa}{r}\PY{l+s+s1}{\PYZsq{}}\PY{l+s+s1}{\PYZdl{}x = }\PY{l+s+si}{\PYZpc{}.2f}\PY{l+s+s1}{ }\PY{l+s+s1}{\PYZbs{}}\PY{l+s+s1}{times10\PYZca{}}\PY{l+s+s1}{\PYZob{}}\PY{l+s+s1}{\PYZhy{}3\PYZcb{}}\PY{l+s+s1}{\PYZbs{}}\PY{l+s+s1}{, }\PY{l+s+s1}{\PYZbs{}}\PY{l+s+s1}{text}\PY{l+s+si}{\PYZob{}см\PYZcb{}}\PY{l+s+s1}{\PYZdl{}}\PY{l+s+s1}{\PYZsq{}} \PY{o}{\PYZpc{}} \PY{p}{(}\PY{n}{distance} \PY{o}{*} \PY{l+m+mi}{10}\PY{o}{*}\PY{o}{*}\PY{l+m+mi}{3}\PY{p}{)}\PY{p}{)}
\end{Verbatim}

\texttt{\color{outcolor}Out[{\color{outcolor}8}]:}
    
    $x = 3.72 \times10^{-3}\, \text{см}$

    


    % Add a bibliography block to the postdoc
    
    
    
    \end{document}
